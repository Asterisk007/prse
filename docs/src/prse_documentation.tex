% Document setup
\documentclass[letterpaper, 12pt]{article}
\addtolength{\oddsidemargin}{-.875in}
\addtolength{\evensidemargin}{-.875in}
\addtolength{\textwidth}{1.75in}
\addtolength{\topmargin}{-.875in}
\addtolength{\textheight}{1.75in}
% Adding a blank page command
\def\blankpage{%
      \clearpage%
      \thispagestyle{empty}%
      \addtocounter{page}{-1}%
      \null%
      \clearpage}
% Packages to include
\usepackage{amsfonts}
\usepackage{amsmath}
\usepackage[document]{ragged2e}
\usepackage{hyperref}
\hypersetup{
   colorlinks = true,
   linkcolor = {blue}
}
\usepackage{mathtools}
\usepackage{tabularx}
\usepackage{makecell}
\usepackage{listings}
\usepackage{xcolor}
% New commands
\newcommand{\Z}{\mathbb{Z}}
\newcommand{\R}{\mathbb{R}}
\definecolor{darkgreen}{RGB}{0, 145, 39}

\renewcommand\theadfont{\bfseries}
\lstset{
   frame=tb,
   language=C++,
   aboveskip=3mm,
   belowskip=3mm,
   columns=flexible,
   basicstyle={\small\ttfamily},
   numbers=none,
   numberstyle=\tiny\color{gray},
   keywordstyle=\color{blue},
   commentstyle=\color{darkgreen},
   stringstyle=\color{mauve},
   breaklines=true,
   breakatwhitespace=true,
   tabsize=3
}
% Document start
\begin{document}
\title{\underline{PRSE}\\ A Programming Language for Tired C++ Developers}
\author{Written by Daniel Ellingson\\with advice and support from Dr. Todd Gibson}
\date{Compiled from {\LaTeX} source on \today}
\maketitle

\newpage

\tableofcontents

\newpage

\section{Introduction}
Having spent four years of my life learning C++ and various other programming languages, I feel that
it is within my level of experience to say that most C-like languages fail to do everything the best
way. C may have its syntax, and C++ may have its classes, and other languages may have their own
features which make them useful for their given paradigm--but I wish for a language that has the
syntax I {\em prefer}. Enter PRSE.\linebreak

PRSE has features that most other C-like languages have, with a few extra goodies on the side for
those who may tire of repeated code.

\begin{itemize}
\item C\#-/Java-style classes, for those who think header files are outdated
\item Python-style dictionaries, for those who like JSON. Yes, we have both JSON and actual classes.
   Both are handy, I assure you
\item R-style vectorization: that is, items in an array or in an STL vector can be easily iterated
   over. More in \hyperref[sec:flowcontrol-loops]{flow control - loops}
\end{itemize}

\newpage

\section{Basic Program Structure}

A PRSE program will usually consist of the following:

\begin{itemize}
\item Zero or more \texttt{use} statements which are followed by a string or list of strings, e.g. \texttt{"io", "math", "algorithm"}
\begin{itemize}
   \item These are used both for STL (standard template) libraries and third-party libraries, such as those created by other users and yourself
\end{itemize}
\item Zero or more variable declaration and/or class declaration statements
\item Zero or more function definitions. 
\end{itemize}

$\left\{
   \begin{tabularx}{\textwidth}{>{\centering\arraybackslash}X}
   {\small From this definition, one may infer that an empty program is just as valid as a full program
   complete with variables, functions, and class definitions. While this is certainly true,
   such a program has no practical operation, and in fact will cause a warning to be printed to
   the terminal.}
   \end{tabularx}
\right\}$\linebreak
\linebreak

\hrulefill

A complete program written using PRSE needs a single \texttt{main} function to
run. Like all functions, this \texttt{main} function can return any primitive data type (see
\hyperref[sec:primitives]{primitives} for more information).

\subsection{The Main Function}

A basic standard main function might look like this:\linebreak

\begin{lstlisting}
function main(ac: int, av: string[]){
   
}
\end{lstlisting}

Main Function Arguments \linebreak

\texttt{ac: int} declares a function argument named \texttt{ac} with primitive type \texttt{int}.
Similarly, \texttt{av: string[]} declares an argument named \texttt{av} with array type \texttt{string}.
\texttt{ac} can be used in the function to find the number of arguments passed to the program when it is run,
and likewise \texttt{av} can be used to obtain the \texttt{string} values of each argument.

\newpage

\section{Basic Data Types}
\subsection{Primitives}
\label{sec:primitives}

PRSE supports most of the same primitive data types that C++ does, save for floating point numbers.
Doubles are supported in place of floating point numbers for the sake of simplicity in programming
the compiler. Refer to \autoref{tab:availablePrimitives} for basic data types that can be used
without additional libraries.\linebreak

\begin{table}[h]
\centering
   \caption{Primitive data types}
   \begin{tabularx}{400pt}{|X|X|X|}
      \hline
      \thead{Data type} & \thead{Supported formats} & \thead{Example(s)}     \\ \hline      
      Integer           & $ x \in \Z $              & 10, 12, -9, 8          \\ \hline
      Boolean           & $ x \in {true, false} $   & \it{true}, \it{false}  \\ \hline
      Character         & All ASCII characters      & A, G, f, \%, *         \\ \hline
      Double            & $ x \in \R $              & 12.45, 6.77777, 2.43   \\ \hline
   \end{tabularx}
   \label{tab:availablePrimitives}
\end{table}

\newpage

\section{Flow Control}

\subsection{Conditionals}

If you're familiar with any C-like language, then you already know how to program an if-statement in
PRSE. However, if this is your first rodeo, here's what to expect:

\begin{lstlisting}
if (/* Some condition, e.g. a >= b */){
   // Execute code
}
\end{lstlisting}

For a more concrete example, suppose we have the following code:

\begin{lstlisting}
let foo: int = 25;
let bar: bool = false;

if (foo > 4 && bar == true){
   foo = 2;
}
\end{lstlisting}

Then the above if-statement will {\em not} execute because \texttt{bar} is false.\linebreak

In addition, we can also use \texttt{else if} and \texttt{else} clauses in a PRSE program to execute
code depending on how the conditions change:

\begin{lstlisting}
if (/* Conditions */){

} else if (/* Other conditions */) {

} else {
   // If all other conditions fail, execute this code.
}
\end{lstlisting}

\subsection{Loops}
\label{sec:flowcontrol-loops}

\end{document}
% Document end
